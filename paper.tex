\documentclass[conference]{IEEEtran}

\ifCLASSOPTIONcompsoc
\usepackage[nocompress]{cite}
\else
\usepackage{cite}
\fi


\usepackage{flushend}


\usepackage[pdftex]{graphicx}
\graphicspath{{./png/}{./eps/}}
\DeclareGraphicsExtensions{.pdf,.jpeg,.png,.eps}

\usepackage{amsmath}
\interdisplaylinepenalty=2500

\hyphenation{op-tical net-works semi-conduc-tor}

\begin{document}
\title{Sacrificing Interpretability for Better Performance in Fuzzy Inference Systems}


\author{\IEEEauthorblockN{Amaury Hernandez-Aguila, Mario
    Garcia-Valdez, Oscar Castillo}
  \IEEEauthorblockA{Tijuana Institute of Technology\\
    Division of Graduate Studies and Graduates\\
    Tijuana, Mexico\\
    Email: \{amherag,mario,ocastillo\}@tectijuana.edu.mx}}

\maketitle

\begin{abstract}

\end{abstract}

\IEEEpeerreviewmaketitle

\section{Throwing References}

A case study to illustrate the use of non-convex membership functions for linguistic terms
\cite{garibaldi2004case}

Interpretability constraints for fuzzy information granulation
\cite{mencar2008interpretability}

Analysis of Fuzzy Entropy and Similarity Measure for Non Convex
Membership Functions
\cite{lee2009analysis}

frbs: fuzzy rule-based systems for classification and regression in R
\cite{riza2015frbs}

A new hybrid enhanced local linear neuro-fuzzy model based on the optimized singular spectrum analysis and its application for nonlinear and chaotic time series forecasting
\cite{abdollahzade2015new}

Optimization of the Fuzzy Integrators in Ensembles of ANFIS Model for
Time Series Prediction: The case of Mackey-Glass
\cite{soto2015optimization}

\section{Introduction}
\label{introduction}

\section{Related Work}
\label{related-work}



\begin{equation}
  \label{squared-sine}
  \begin{aligned}
    sin^2(x)
\end{aligned}
\end{equation}



\section{Preliminaries}
\label{preliminaries}

\section{Proposed Method}
\label{proposed-method}

% \begin{figure}[!t]
%   \centering
%   \includegraphics[width=2.5in]{fs-as-ifs}
%   \caption{Classic Fuzzy Set Represented as an Intuitionistic Fuzzy Set}
%   \label{fs-as-ifs}
% \end{figure}

\section{Experiments}
\label{experiments}

\begin{enumerate}
  \item if $t_{0}$ is $A_{1}$ then $p_{t+1}$ is $C_{1}$
  \item if $t_{-6}$ is $A_{2}$ then $p_{t+1}$ is $C_{2}$
  \item if $t_{-12}$ is $A_{3}$ then $p_{t+1}$ is $C_{3}$
  \item if $t_{-18}$ is $A_{4}$ then $p_{t+1}$ is $C_{4}$
\end{enumerate}

\section{Results}
\label{results}

\begin{table}[!t]
  \renewcommand{\arraystretch}{1.3}
  \caption{Hypothesis Tests for the Training Stage, First Experiment}
  \label{first-hypothesis-tests-training}
  \centering
  \begin{tabular}{|c|c|c|c|c|c|}
    \hline
    Method & $\mu_{TR}$ & $SD_{TR}$ & $n$ & t-Value & CI \\
    \hline
    T1-IFIS & 89.50 & 25.36 & 60 &  & \\
    \hline
    T1-FIS & 108.05 & 23.47 & 60 & -4.1584 & $>$ 99.9\% \\
    \hline
    IT2-FIS (\(\mu\)) & 112.56 & 29.16 & 100 & -5.2597 & $>$ 99.9\% \\
    \hline
    IT2-FIS (SD) & 102.31 & 26.08 & 300 & -3.5548 & $>$ 99.9\% \\
    \hline
  \end{tabular}
\end{table}

\section{Conclusion}
\label{conclusion}


\section{Future Work}
\label{future-work}


\bibliographystyle{IEEEtran}
\bibliography{paper}

\end{document}
